\section{Introduction}
Introduction:
Federated Learning~\cite{li2020federated}.

\subsection{Related Surveys}
In recent years, federated learning has become a buzzword in various fields, leading to the emergence of numerous FL studies.
These works can be classified into two primary categories: FL system design and FL appllications. 
The initial architectures and concepts for FL systems were summaried by Yang \textit{et al.}~\cite{yang2019federated}. 
They categorize FL into horizontal FL, vertical FL and federated transfer learning based on the distribution characteristics of data, 
which are written in IEEE Standard 3652.1-2020~\cite{yang2021white, IEEEstd3652}. 
Following this, an increasing number of surveys have emerged focusing on enhancing FL system design. 
From the algorithmic perspective, personlized FL~\cite{tan2022towards, kulkarni2020survey} aims to learn personlized models for each client to address the challenge of statistical heterogeneity.
Besides, the privacy-perserving model aggregation protocols of FL systems also been widely studied and sumaried by~\cite{liu2022privacy,el2022differential,yin2021comprehensive,lyu2020threats}
Furthermore, Many advance FL architectures had been proposed and summaried, such as Decent
Some surveys 



\subsection{Distinction of Our Survey}
However,

Federated leanring is promising 

\begin{comment}
    ssssdd

\end{comment}