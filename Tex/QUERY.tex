\section{Query-based Federated Learning}
\label{sec:query}
\subsection{Overview}
Let us continue by establish a sustainable open FL platform based on a query-based cooperation framework.
An overview of this platform is presented in Fig.~\ref{fig:query}, the desin philosophy behind this framework is to break the coupling between FL server and clients.
In the query-based FL systems, all traditional FL roles and components are maintained on an open model repository called Model Community. The Model Community privdes a one-stop ML models redistribution and reuse service, including model indexing, automatic batch model reuse, license management, privacy control and so on.
In addition to large-scale pretrained models like BERT~\cite{devlin2018bert}, BLOOM~\cite{scao2022bloom} with great generalization abilities, we also encourage individuals to upload their task-specific models trained on limited domain data to boost the knowledge mining within models~\cite{you2021workshop}.
The derivatives of knowledge mining can learn representations from multiple domains, resulting in more promising performance that can be evaluated by platform users.
Furthermore, the contributors can open models under applicable licenses, granting them distribution control and legal protection of their intellectual property.
In summary, the properties of query-based FL are:
(1) \textbf{Model Agnostic}, as there are no restrictions on the types and architectures of the models uploaded by users;
(2) \textbf{Contactless}, as coomunication channels need not be maintained; 
(3) \textbf{Community-powered}, whereby sharing models enrichs the entire community.

Actually, we aim to advocate a novel SaaS~\cite{brereton1999future} ML platform with automatic model reuse integrated, which has potential to leverage the transportability of models to address previously unexplored ML problems.
Due to the high computational demands of deep learning, current ML platforms primarily concentrate on computing, for example, MaaS, MLaaS, FLaaS provide ML models deployment and development services to handle user-specified tasks. (Section~\ref{sec:aas}).
On the other hand, there are several ML platforms provide open model search and download services. 
So, can we leverage leverage off-the-shelf open model platforms to build a query-based FL system?
Unfortunately, these platforms are designed solely for sharing and are no suitable for more advanced functionalities such as model ensemble~\cite{jacobs1991adaptive} and knowledge distillation~\cite{hinton2015distilling}, we will explain the reasons in the following section.

% Model Mining

\begin{figure}[t]
  \centering
  \includegraphics[width=\linewidth]{fig/query_frame.pdf}
  \caption{An overview of query-based FL systems. (U: model User, C: Coordinator, D: Data owner, A: Auditor)}
  \Description{}
  \label{fig:query}
\end{figure}

\subsection{How to Query for Models}
To establish a query-based FL platform, the first thing that comes to mind is how to query for models.
Unlike traditional ML model sharing repositories that mainly query for a specific model by name, it requires an efficiency approach to export a batch of target models that ready for ensemble or distillation.
We summaried the filter conditions of existing DNNs sharing repositories in Table.~\ref{table:repository}.
The prevailing method for querying models involves searching for the desired model by its name, datasets used, associated tasks.
To illustrate, one might search for the model name GPT~\cite{radford2019language}, models trained on the MNIST dataset~\cite{lecun2010mnist}, or models capable of performing image segmentation tasks.
However, this model retrieval method requires the users have a strong priori knowledge in data science, thus raising the barrier for knowledge mining within models.
For example, there is no effective way to acquire a batch of image classfication models that contains the knowledge of \textit{lesser panda} for further distillation.
A compromise solution is to manually search the schema of each dataset one-by-one and subsequently search for models trained on those datasets.

Additionally, as shown in Table.~\ref{table:repository}, most DNNs repositories are simply list the description of input/output (e.g., NVIDIA NGC, OpenVINO) or even just present the source codes (e.g., Tensorflow Hub, Pytorch Hub),
This lack of unified convention for model input/ouput poses a challenge for query-based FL.
Besides, most of DNNs repositories do not enable querying models by licenses, resulting in the cumbersome task of individually handling model licenses and ensuring compatibility among different licenses.
Hence, it is imperative to reconsider the design of DNN repositories to enable quick identification of readily reusable models for model knowledge mining. We further suggest following filter conditions for query-based FL.

\begin{table}[t]
  \caption{Filter conditions and characteristics of DNNs repositories. \checkmark : Supported, \ding{55} : Unsupported, \textbf{!} : Information provided but unsearchable, listed in descending order by number of models.}
  \label{table:repository}
  \footnotesize
  \begin{tabular}{|l|c|c|c|c|c|c|c|c|}
  \hline
  & \multicolumn{1}{l|}{DS Name} & \multicolumn{1}{l|}{Model Architecture} & \multicolumn{1}{l|}{Modality/Task} & \multicolumn{1}{l|}{Tag} & \multicolumn{1}{l|}{License} & \multicolumn{1}{l|}{Input-Output} & \multicolumn{1}{l|}{Batch Export} & \multicolumn{1}{l|}{\# of Models}\\ \hline
  Hugging Face\tablefootnote{https://huggingface.co}
  & \checkmark & \checkmark & \checkmark & \checkmark & \checkmark & \textbf{!} & \ding{55} & 133,641 \\ \hline
  Model Zoo\tablefootnote{https://modelzoo.co/} & \checkmark & \checkmark & \checkmark & \checkmark & \ding{55} & \ding{55} & \ding{55} & 3,426 \\ \hline
  Tensorflow Hub\tablefootnote{https://tfhub.dev/}& \checkmark & \checkmark & \checkmark & \checkmark & \textbf{!} & \textbf{!} & \ding{55} & 1,356 \\ \hline
  NVIDIA NGC\tablefootnote{https://catalog.ngc.nvidia.com/models} & \textbf{!} & \checkmark & \checkmark & \checkmark & \textbf{!} & \textbf{!} & \ding{55} & 527 \\ \hline
  OpenVINO\tablefootnote{https://docs.openvino.ai/latest/model\_zoo.html} & \textbf{!} & \checkmark & \checkmark & \ding{55} & \textbf{!} & \textbf{!} & \checkmark & 278 \\ \hline
  Pytorch Hub\tablefootnote{https://pytorch.org/hub/} & \textbf{!} & \checkmark & \ding{55} & \ding{55} & \ding{55} & \textbf{!} & \ding{55} & 49 \\ \hline
  \end{tabular}
\end{table}

\subsubsection{Data Description} % 数据的domain和统计特征,数据的正确性,如何打标签,
Similar with the data heterogeous challenges in FL~\cite{li2021federated}. The local datasets of contributors have varing quality and contain intractable biases, imbalances and noisies that can be attributed to the natural characteristics of demographic or improper data collection mechanisms~\cite{dayan2021federated}.
Besides, label errors pervasive even in open datasets~\cite{northcutt2021pervasive}. 
So, in addition to searching for domain-specific datasets based on their data descriptions, we are also seeking such descriptions for the purpose of future traceability and debugging.
The data description can consist of statistical analysis results or the visualization diagrams that used to profile the data distribution~\cite{li2020multi} and complementary provenance information.

\subsubsection{Workflow and History}
The process of building an ML model is iterative, involving repeated hyperparameter tuning and architecture exploration, resulting in abundant workflow and historical trajectory data.
This information includes pipelines, model structures, hyperparameter values for pre-training and fine-tuning, test metrics, and results. 
These data can be useful in filtering models that meet specific requirements, such as those with data standardization in preprocessing or evaluated using mean average precision (mAP).
Instead of manually saving and uploading the logs and configuration files, a more efficient method is to leverage ML workflow management tools~\cite{vartak2016modeldb}, such as MLflow~\footnote{https://mlflow.org} and Neptune~\footnote{https://neptune.ai}, to automatically track and store the ML workflow during model building process. 
Additionally, to ensure that the computational consumption of models is within budget, the Deep Learning Profiler~\footnote{https://docs.nvidia.com/deeplearning/frameworks/dlprof-user-guide/index.html} can be leveraged to generate a report that shows the FLOPS and bandwidth requirements.

\subsubsection{Software Dependency}
ML models are software that depend on underlying ML libraries, so it is important to declare the dependencies of the model to analyze software compatibility between batches of models. 
For instance, resource-constrained devices may need to trim down the list of software-dependent libraries to meet limited storage space requirements~\cite{david2021tensorflow}.
In some cases, contributed models may rely on other models as dependencies. 
For example, Fast R-CNN~\cite{girshick2015fast} uses VGG16~\cite{simonyan2014very} as its backbone. 
It is crucial to release this information for further model license compatibility analysis.

The aforementioned filter conditions provide comprehensive coverage of the ML modeling process. 
However, there are additional requirements depending on the reuse mechanisms of the model retrieval side. 
For example, FedAvg~\cite{mcmahan2017communication} aggregates the local models weights element-wise, which requires full access to the models. 
In contrast, MoE with a gating network~\cite{jacobs1991adaptive} only ensembles a batch of model outputs, so the individual models can remain blackboxes in this scenario.
So, in the context of software licenses or model licenses, the batch models reused by FedAvg should be released as source code, while those reused by MoE can be released as binary object code (static linking).
The above distinction is critical for ensuring that model reuse results meet the legal framework, and this has been overlooked in traditional FL.
We will expand on this topic in the following section.

\subsection{How to Reuse Batch of Models}
Once we have acquired a certain number of models that can contribute to the new target task, the next step is to reuse the knowledge of these pre-trained models, i.e., transfer their knowledge from source domain to the target domain~\cite{pan2009survey}.
However, before deciding on how to reuse the model, it is important to ensure that the necessary legal rights and permissions have been obtained. 
This may involve reviewing the terms and conditions of the licenses under which the models were originally released or obtaining permission from the original creators or copyright holders.
Therefore, in this section, we will not focus on the technical details of how to reuse models, which is already covered by many related surveys, such as Transfer Learning~\cite{pan2009survey}, Ensemble Leanring~\cite{zhou2012ensemble}, Domain Adaptation~\cite{wang2018deep}, Knowledge Distillation~\cite{wang2021knowledge}, Deep Generative Models~\cite{cao2022survey} and Model Fusion~\cite{ji2021emerging}.
Furthermore, the specific model reuse technique or techniques used is at the user's discretion, and the query-based FL platform we advocate is not bound or restricted to any particular model reuse algorithm.
Innovatively, we study how to reuse batch of models, from the perspective of \textbf{legal compliance}.
%Therefore, the focus of this section is how to reuse batch of models, from the perspective of legal compliance.

The machine learning community benefits from the openness of ideas and code, and many high-impact ML conferences and journals encourage authors to publish their source code and dataset to research platforms like Papers With Code~\footnote{https://paperswithcode.com} and Code Ocean~\footnote{https://codeocean.com} to increase exposure and facilitate reproducibility.
To restrict the use of ML techniques for unethical purposes (i.e. Deepfakes~\cite{mirsky2021creation}) and protect the intellectual property (IP) of creators, models are typically published under a license agreed upon by the licensor.
Here, we summary the licenses, granted rights, restrictions and enforcements for ML models posted on Hugging Face in Table.~\ref{tab:licenses}.

\subsubsection{Model Licensing Forms}
ML models are licensed in three main forms: as software (e.g. Apache, MIT, GPL), as a model (e.g. OpenRAIL), and as content/database (e.g. CC-BY, PDDL).
The reason for the mixed use of licenses is the ambiguity in the dependency relationship between the code, model, and data.
ML models can be released with reproducable code and be considered as a component of software.
So many open software licenses are naturally deferred for licensing of models.
The most popular license is Apache-2.0, which is a permissive open software license that allows the freedom to make derivative works.
However, the model building process also relies on a massive amount of data~\cite{lecun2015deep} that may be licensed under different licenses, which can lead to license conflicts.
A practical example is BERT~\cite{devlin2018bert}, which was published under the Apache-2.0 license but pre-trained on English Wikipedia documents that are licensed under CC BY-SA 3.0.
This changing of license violates the requirement of the CC BY-SA 3.0, which states that any contribution must be distributed under the \textbf{same license} as the original work.

%We don’t claim ownership of the content you create with GPT-2, so it is yours to do with as you please. We only ask that you use GPT-2 responsibly and clearly indicate your content was created using GPT-2.

From the perspective of content and database licensing, some word embedding models, such as GloVe~\cite{pennington2014glove}, compute vector representations of words based on licensed open linguistic resources.
These representations can be regarded as a translation of corpus and fall under the license of the original linguistic resources.
A more complex scenario arises when the model is fine-tuned with other data that has a different license, for example, fine-tune RoBERTa~\cite{liu2019roberta} (MIT license) with SQuAD2~\cite{rajpurkar2016squad} (CC BY 4.0).
The resulting model can be interpreted as both derived works and combined works.

Not only limited to protecting the intellectual property and controlling the diffusion of ideas, but AI companies and researchers are also concerned about licensees using their models for unethical purposes~\cite{jobin2019global, awad2018moral, yuste2017four}, which is not restricted by traditional licenses based on the context of software and content.
We can infer the concerns of the inventors of GPT-2~\cite{radford2019language} about the unethical use of the model from its modified MIT license, which states, \textit{We don't claim ownership of the content you create with GPT-2, so it is yours to do with as you please. We only ask that you use GPT-2 responsibly and clearly indicate your content was created using GPT-2.} 
However, such a statement lacks legal enforcement, and users may avoid accountability by convincing themselves that despite their efforts to minimize harm, they could not predict the AI artifact they generated would be used for harmful purposes.
Besides, the original licensing frameworks (e.g. MIT, CC BY) for software and content are not well suited to the data-driven ML. 
Many ML operations, such as training, fine-tuning, inference, and distillation, are not explicitly defined in traditional software and content licenses, leaving a potential legal loophole for licensees.

To address the unique challenges and considerations surrounding the use and distribution of ML models, several specific licenses for ML models have been proposed. 
The CreativeML OpenRAIL-M license, proposed by Responsible AI~\cite{contractor2022behavioral}, is the most popular model-specific license on Hugging Face and enables legally enforceable responsible use.
By accepting this license, licensees must adhere to the use-based restrictions stated by the licensor, and these restrictions must also apply to derivative works.
With a multitude of different model licenses available, it becomes a challenging and tedious task to reuse them in bulk. 
It is therefore imperative to establish guidelines for selecting a license for models that are ready for query-based FL.

\subsubsection{Model License Choosing Preferences}
In query-based FL, the model community automatically reuses models contributed by users, which raises unique concerns about licensing rights.
Firstly, the license should allow the modification, combination and redistribution of the works and any derived works.
Secondly, the sublicensing right can lubricate the republication of derived works resulting from knowledge mining.
Thirdly, some licenses require the source of the derived works to be disclosed and prohibit their commercial use, which hinders model selling~\cite{chen2019towards}.
Lastly, some licenses are copyleft (denoted by * in Table.~\ref{tab:licenses}), which means the derivatives should be licensed under the same license or a compatible one, leading to potential license conflicts and license proliferation~\cite{gomulkiewicz2009open}.

In summary, the 

Another example is finetune the pretrained model RoBERTa~\cite{liu2019roberta} (MIT license) with SQuAD2~\cite{rajpurkar2016squad} licensed under CC-BY-4.0.
%许可扩散问题

%LEAGLE-BERT 的问题
However, BERT~\cite{devlin2018bert} pretrained corpus using English Wikipedia document CC BY-SA 3.0 no compatible with Apache-2.0

difussion of ideas

The 

Therefore, the focus of this section is how to reuse batch of models, 

There are many efforts deal with model reuse.

Single Model Reuse

Batch Model Reuse


\begin{table}[t]
  \centering
  \scriptsize
  \caption{Licenses for ML models available on Hugging Face with a focus on their rights, restrictions and enforcements, grouped by free software licenses, AI model licenses, free content or database licenses in descending order of number of models (GPL, BSD, LGPL, CC licenses with unspecified versions are excluded, the similar revisions are merged). \checkmark : Permited or Required, \ding{55} : Not Permited or Not Required, \textbf{!} : Not Explicitly Permited, * : Copyleft License}
  \label{tab:licenses}
  \begin{tabular}{r||ccc|ccc|cccc|c|p{3.5cm}}
    \toprule
    Licenses
    & \multicolumn{1}{P{90}{2.0cm}}{Modify / Merge} &
      \multicolumn{1}{P{90}{2.0cm}}{Redistribution} &
      \multicolumn{1}{P{90}{2.0cm}}{Sublicensing} & 
      \multicolumn{1}{P{90}{2.0cm}}{Commercial Use} & 
      \multicolumn{1}{P{90}{2.0cm}}{Patent Use} & 
      \multicolumn{1}{P{90}{2.0cm}}{Trademark Use} &
      \multicolumn{1}{P{90}{2.0cm}}{State Changes} &
      \multicolumn{1}{P{90}{2.0cm}}{Disclose Source} &
      \multicolumn{1}{P{90}{2.0cm}}{Responsible-use Restrictions} &
      \multicolumn{1}{P{90}{2.0cm}}{License/Disclaim Preservation} &
      \multicolumn{1}{P{90}{2.0cm}}{\# of Models} &
      \multicolumn{1}{c}{Licensed Materials / Remarks}    \\
    \midrule
    Apache-2.0 & \checkmark & \checkmark & \checkmark & \checkmark & \checkmark & \ding{55} & \checkmark & \ding{55} & \ding{55} & \checkmark & 23,519 & BERT~\cite{devlin2018bert} \\
    MIT &  \checkmark & \checkmark & \checkmark & \checkmark & \textbf{!} & \textbf{!} & \ding{55} & \ding{55} & \ding{55} & \checkmark & 9,605 & GPT-2~\cite{radford2019language} \\
    AFL-3.0 & \checkmark & \checkmark & \checkmark & \checkmark & \checkmark & \ding{55} & \checkmark & \ding{55} & \ding{55} & \checkmark & 1,561 & Italian-Legal-BERT~\cite{licari2022italian} \\
    *GPL-3.0 & \checkmark & \checkmark & \ding{55} & \checkmark & \checkmark & \ding{55} & \checkmark & \checkmark & \ding{55} & \checkmark & 404 & CKIP BERT Chinese \\
    Artistic-2.0 & \checkmark & \checkmark & \checkmark & \checkmark & \checkmark & \ding{55} & \checkmark & \ding{55} & \ding{55} & \checkmark & 331 & Include original source \\
    BSD-3-Clause\&-Clear & \checkmark & \checkmark & \checkmark & \checkmark & \textbf{!} & \textbf{!} & \ding{55} & \ding{55} & \ding{55} & \checkmark & 209 & CodeGen~\cite{nijkamp2022conversational} / A MIT-style license \\
    WTFPL-2.0 & \checkmark & \checkmark & \textbf{!} & \checkmark & \textbf{!} & \textbf{!} & \ding{55} & \ding{55} & \ding{55} & \ding{55} & 131 & A MIT-style permissive license  \\
    *AGPL-3.0 & \checkmark & \checkmark & \ding{55} & \checkmark & \checkmark & \ding{55} & \checkmark & \checkmark & \ding{55} & \checkmark & 96 & Distributed under AGPL only  \\
    Unlicense & \checkmark & \checkmark & \textbf{!} & \checkmark & \textbf{!} & \textbf{!} & \ding{55} & \ding{55} & \ding{55} & \ding{55} & 90 & A MIT-style permissive license  \\
    %GPL & 1 & 2 & 3 & 4 & 5 & 6 & 7 & 8 & 9 & 10 & 63 &  \\
    BSL-1.0 & \checkmark & \checkmark & \checkmark & \checkmark & \textbf{!} & \textbf{!} & \ding{55} & \ding{55} & \ding{55} & \checkmark & 60 & A MIT-style permissive license \\
    %BSD & 1 & 2 & 3 & 4 & 5 & 6 & 7 & 8 & 9 & 10 & 43 &  \\
    *GPL-2.0 & \checkmark & \checkmark & \ding{55} & \checkmark & \textbf{!} & \textbf{!} & \checkmark & \checkmark & \ding{55} & \checkmark & 34 & Not compatible with GPL-3.0  \\
    BSD-2-Clause & \checkmark & \checkmark & \checkmark & \checkmark & \textbf{!} & \textbf{!} & \ding{55} & \ding{55} & \ding{55} & \checkmark & 34 & A MIT-style permissive license  \\
    *LGPL-2.1\&3.0 & \checkmark & \checkmark & \ding{55} & \checkmark & \textbf{!} & \textbf{!} & \checkmark & \checkmark & \ding{55} & \checkmark & 25 & For software libraries  \\
    *OSL-3.0 & \checkmark & \checkmark & \checkmark & \checkmark & \checkmark & \ding{55} & \checkmark & \checkmark & \ding{55} & \checkmark & 22 & Linking is not derivative work \\
    %BSD-3-Clause-Clear & 1 & 2 & 3 & 4 & 5 & 6 & 7 & 8 & 9 & 10 & 14 &  \\
    %LGPL & 1 & 2 & 3 & 4 & 5 & 6 & 7 & 8 & 9 & 10 & 12 &  \\
    ECL-2.0 & \checkmark & \checkmark & \checkmark & \checkmark & \checkmark & \ding{55} & \checkmark & \ding{55} & \ding{55} & \checkmark & 12 & For education communities \\
    *MPL-2.0 & \checkmark & \checkmark & \checkmark & \checkmark & \checkmark & \ding{55} & \checkmark & \checkmark & \ding{55} & \checkmark & 9 & State changes under MPL only  \\
    ISC & \checkmark & \checkmark & \textbf{!} & \checkmark & \textbf{!} & \textbf{!} & \ding{55} & \ding{55} & \ding{55} & \checkmark & 8 & MIT-style license w/o sublicense \\ % Permission to distribute this software for any purpose
    Zlib & \checkmark & \checkmark & \textbf{!} & \checkmark & \textbf{!} & \textbf{!} & \ding{55} & \ding{55} & \ding{55} & \checkmark & 8 & Rename if modified \\
    *Ms-PL & \checkmark & \checkmark & \checkmark & \checkmark & \checkmark & \ding{55} & \ding{55} & \ding{55} & \ding{55} & \checkmark & 7 & Weak copyleft license \\ % Weak copyleft
    *EPL-1.0\&2.0 & \checkmark & \checkmark & \checkmark & \checkmark & \checkmark & \textbf{!} & \ding{55} & \checkmark & \ding{55} & \checkmark & 6 & Can link proprietary license code \\
    NCSA & \checkmark & \checkmark & \checkmark & \checkmark & \textbf{!} & \ding{55} & \ding{55} & \ding{55} & \ding{55} & \checkmark & 4 & Include full text of license \\
    PostgreSQL & \checkmark & \checkmark & \textbf{!} & \checkmark & \textbf{!} & \textbf{!} & \ding{55} & \ding{55} & \ding{55} & \checkmark & 2 & A MIT-style license \\
    OFL-1.1 & \checkmark & \checkmark & \ding{55} & \checkmark & \textbf{!} & \textbf{!} & \ding{55} & \ding{55} & \ding{55} & \checkmark & 2 & For font software \\
    %EPL-1.0 & 1 & 2 & 3 & 4 & 5 & 6 & 7 & 8 & 9 & 10 & 2 &  \\
    *EUPL-1.1 & \checkmark & \checkmark & \checkmark & \checkmark & \checkmark & \ding{55} & \checkmark & \checkmark & \ding{55} & \checkmark & 1 & License of EU covers SaaS \\
    %LGPL-2.1 & 1 & 2 & 3 & 4 & 5 & 6 & 7 & 8 & 9 & 10 & 1 &  \\
    LPPL-1.3c & \checkmark & \checkmark & \checkmark & \checkmark & \textbf{!} & \ding{55} & \checkmark & \checkmark & \ding{55} & \checkmark & 1 & Covering  stewardship transfer \\
    
    \hline
    %\textbf{Model Licenses $\downarrow$} & \multicolumn{12}{l}{} \\
    \hline
    
    CreativeML-OpenRAIL-M & \checkmark & \checkmark & \checkmark & \checkmark & \checkmark & \ding{55} & \checkmark & \ding{55} & \checkmark & \checkmark & 3,590 & Stable Diffusions v1~\cite{rombach2022high} \\
    OpenRAIL &  \multicolumn{10}{l|}{>Responsible AI License template, w/o full text} & 2,393 & ControlNet~\cite{zhang2023adding}  \\
    BigScience-BLOOM-RAIL-1.0 & \checkmark & \checkmark & \checkmark & \checkmark & \checkmark & \ding{55} & \checkmark & \ding{55} & \checkmark & \checkmark & 196 & BLOOM~\cite{scao2022bloom} \\
    BigScience-OpenRAIL-M & \multicolumn{10}{l|}{>Same as BigScience-BLOOM-RAIL-1.0} & 155 & A general version of 1.0 \\
    OpenRAIL++ & \multicolumn{10}{l|}{>Same as CreativeML-OpenRAIL-M} & 72 & Stable Diffusion v2~\cite{rombach2022high} \\
    OPT-175B & \checkmark & \ding{55} & \ding{55} & \ding{55} & \ding{55} & \ding{55} & \ding{55} & \ding{55} & \checkmark & \checkmark & $\approx66$ & OPT LLM~\cite{zhang2022opt} \\
    SEER &  \multicolumn{10}{l|}{>Same as OPT-175B, ban on reverse-engineer} & / & SEER Vision Model~\cite{goyal2022vision} \\
    
    \hline
    \hline

    CC-BY-4.0\&3.0\&2.5\&2.0 & \checkmark & \checkmark & \ding{55} & \checkmark & \ding{55} & \ding{55} & \checkmark & \ding{55} & \ding{55} & \checkmark & 1,740 & RoBERTa-SQuAD2.0~\cite{rajpurkar2016squad} \\
    *CC-BY-SA-4.0\&3.0 &  \checkmark & \checkmark & \ding{55} & \checkmark & \ding{55} & \ding{55} & \checkmark & \checkmark & \ding{55} & \checkmark & 590 & LEGAL-BERT~\cite{chalkidis2020legal} \\
    *CC-BY-NC-SA-4.0\&3.0 & \checkmark & \checkmark & \ding{55} & \ding{55} & \ding{55} & \ding{55} & \checkmark & \checkmark & \ding{55} & \checkmark & 556 & LayoutLMv3~\cite{huang2022layoutlmv3} \\
    CC-BY-NC-4.0\&3.0\&2.0 & \checkmark & \checkmark & \ding{55} & \ding{55} & \ding{55} & \ding{55} & \checkmark & \ding{55} & \ding{55} & \checkmark & 499 & GALACTICA~\cite{taylor2022galactica} \\
    CC0-1.0 & \checkmark & \checkmark & \ding{55} & \checkmark & \ding{55} & \ding{55} & \ding{55} & \ding{55} & \ding{55} & \ding{55} & 165 & BlueBERT~\cite{peng2019transfer} \\
    CC-BY-NC-ND-4.0\&3.0 & \checkmark & \ding{55} & \ding{55} & \ding{55} & \ding{55} & \ding{55} & \ding{55} & \ding{55} & \ding{55} & \checkmark & 21 & NonCommercial, NoDerivatives \\
    PDDL & \checkmark & \checkmark & \ding{55} & \checkmark & \ding{55} & \ding{55} & \ding{55} & \ding{55} & \ding{55} & \ding{55} & 16 & Database-specific license \\
    C-UDA & \checkmark & \checkmark & \checkmark & \ding{55} & \textbf{!} & \textbf{!} & \ding{55} & \ding{55} & \checkmark & \checkmark & 13 & Data for computational use only \\
    *LGPL-LR & \checkmark & \checkmark & \ding{55} & \checkmark & \textbf{!} & \textbf{!} & \checkmark & \checkmark & \ding{55} & \checkmark & 12 & LGPL for linguistic resources \\ %如果仅仅发布embedding模型,那么属于“使用语言资源的作品”,如果包含了语言资源或者加密后的资源,那么属于“使用语言资源的衍生物”,包含在此license范围
    *GFDL &  \multicolumn{10}{l|}{>Same as GPL, a free document license} & 12 & txtai-wikipedia \\
    CC-BY-ND-4.0 & \checkmark & \ding{55} & \ding{55} & \checkmark & \ding{55} & \ding{55} & \checkmark & \ding{55} & \ding{55} & \checkmark & 11 & Disallow making derivatives \\
    ODC-By & \checkmark & \checkmark & \ding{55} & \checkmark & \ding{55} & \ding{55} & \ding{55} & \ding{55} & \ding{55} & \checkmark & 7 & Database license w/o sublicense \\
    *ODbL & \checkmark & \checkmark & \ding{55} & \checkmark & \ding{55} & \ding{55} & \checkmark & \checkmark & \ding{55} & \checkmark & 6 & Automatic relicensing \\
    \bottomrule
  \end{tabular}
\end{table}

%The effect of copyleft-style behavioral-use clauses spreads the requirement from the original licensor on his/her wish and trust on the responsible use of the licensed artifact. This is why OpenRAILs require downstream adoption of the use-based restrictions by subsequent re-distribution and derivatives of the AI artifact, as a means to dissuade users of derivatives of the AI artifact from misusing the latter.

%OPT-175B/SEER LICENSE is not a copyleft license, as it does not require derivative works to be licensed under the same license or a compatible one. It is a proprietary license that allows users to use and reproduce the licensed models subject to certain restrictions.

\begin{figure}[b]
    \centering
    \includegraphics[width=\linewidth]{fig/flowchart.pdf}
    \caption{flowchart}
    \Description{}
    \label{fig:flowchart}
\end{figure}

Model type?

Model heterogeneity?

Black box, White box, Mix?

With val?

Horizontal or Vertical?

Query syntax: Table.~\ref{table:repository}, data description, workflow metadata/history of ML pipeline (Scientific workflow management), model performance and profile (task-specific), software dependency,
model use mode

CreativeML Open RAIL-M:
we added use-based restrictions not permitting the use of the Model in very specific scenarios, in order for the licensor to be able to enforce the license in case potential misuses of the Model may occur.

\subsection{How to Protect Models}