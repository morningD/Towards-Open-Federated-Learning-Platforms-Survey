\section{Related Surveys}
\label{sec:related}
Federated learning has become a buzzword in various fields, leading to the emergence of numerous FL studies.
These works can be classified into three primary categories: FL systems design, FL applications and FL toolkits. Extensive surveys are available to summarized the advancement of federated learning, as shown in Table~\ref{table:surveys}.
The initial architectures and concepts for FL systems were summaried by Yang \textit{et al.}~\cite{yang2019federated}. 
They categorized FL into horizontal FL, vertical FL and federated transfer learning based on the distribution characteristics of data, 
which are written in IEEE Standard 3652.1-2020~\cite{yang2021white, IEEEstd3652}. 
Following this, an increasing number of surveys have emerged focusing on enhancing FL system design~\cite{li2020federated,aledhari2020federated, kairouz2021advances, zhang2021survey, li2021survey}. 
From the algorithmic perspective, personlized FL~\cite{kulkarni2020survey, tan2022towards} aims to learn personlized models for each client to address the challenge of statistical heterogeneity~\cite{ma2022state}.
Besides, the privacy-perserving computing platforms and model aggregation protocols for FL systems also been widely studied and summarized by~\cite{liu2022privacy,el2022differential,yin2021comprehensive,lyu2020threats}.
Furthermore, many advanced FL architectures had been proposed, such as asynchronous~\cite{xu2023asynchronous}, decentralized and blockchain-based FL frameworks~\cite{nguyen2021federated, qu2022blockchain, zhu2022blockchain}.
Given that federated learning technologies enable collaboration among distributed participants in model training and decision-making, this capability holds great promise in a wide range of application scenarios.
For instance, multiple geogrphically distributed medical insitutions can enhace medication recommendation, drug-drug interaction prediction and medical image analysis in a collaborative manner without exchanging any sensitive data~\cite{xu2021federated, pfitzner2021federated, antunes2022federated, rieke2020future}. 
The massive real-time data generated by IoT devices in smart cities~\cite{zhang2022federated, ramu2022federated}, industries~\cite{boopalan2022fusion}, vehicles~\cite{du2020federated} has also sparked interest in exploring how FL technology can be used to deliver more advanced services such as intrusion detection, anomaly detection, fraud detection and network load prediction~\cite{agrawal2022federated, alazab2021federated, ghimire2022recent}.

As summarized in Table~\ref{table:surveys}, most surveys extensively discuss the challenges of efficiency, heterogeneity, privacy in FL systems design, while the surveys from blockchain fields offer the most comprehensive review.
However, except for a few blockchain-based FL studies, most of the listed surveys just present the same story from slightly different angles and backgrounds, i.e., a server sets the model training task and delegates it to data holders to complete. 
This \textit{server-dominated} cooperation framework is a narrow implementation of the FL systems.
Therefore, this survey aims to fill the gap by investigating and surveying the associated tenchnologies that support more open and inclusive cooperation frameworks in FL systems, where all entities, whether they own the data or not, can benefit from it. 
The challenges investigated in this survey are not listed in the Table~\ref{table:surveys}, and to the best of our knowledge, this is the first survey that focuses on the \textbf{cooperation frameworks} of FL.

\textbf{Distinction of Our Survey}.
This survey focuses on exploring the innovative cooperation frameworks in FL, which involves some FL concepts such as decentralized FL, blockchain-based FL, few-shot FL, ML related platforms and services but goes beyond them.
In the following sections, we will distinguish our survey by highlingting the similarities and differences between these related concepts.
%, we will differentiate this survey from other related concepts in the field of FL.


\begin{table}[t]
    \footnotesize
    \caption{Summary of existing FL surveys. SYS denotes FL Systems Design, APP denotes FL Applications, SDC denotes Server-Dominated Cooperation frameworks.
    \cmark: elaborated, \xmark: not elaborated.}
    \label{table:surveys}
    \begin{tabular}{|l|l|lllll|lll|}
    \hline
                       & \multicolumn{1}{c|}{} & \multicolumn{5}{c|}{Challenges}                                                                            & \multicolumn{3}{c|}{Contents}                            \\ \hline
                       Scenarios/Tasks &           FL Surveys            & \multicolumn{1}{l|}{Efficiency} & \multicolumn{1}{l|}{Heterogeneity} & \multicolumn{1}{l|}{Privacy} & \multicolumn{1}{l|}{Incentive} & Decentralize & \multicolumn{1}{l|}{SYS} & \multicolumn{1}{l|}{APP} & SDC \\ \hline
    \multirow{14}{*}{General} &      Yang \textit{et al.}~\cite{yang2019federated}         & \multicolumn{1}{c|}{ \cmark } & \multicolumn{1}{c|}{\cmark} & \multicolumn{1}{c|}{\cmark} & \multicolumn{1}{c|}{\cmark} & \multicolumn{1}{c|}{\cmark}  & \multicolumn{1}{c|}{\cmark} & \multicolumn{1}{c|}{\cmark} & \multicolumn{1}{c|}{\cmark}  \\ \cline{2-10}              
                        &   Li \textit{et al.} 2020~\cite{li2020federated}                    & \multicolumn{1}{c|}{\cmark} & \multicolumn{1}{c|}{\cmark} & \multicolumn{1}{c|}{\cmark} & \multicolumn{1}{c|}{\xmark} & \multicolumn{1}{c|}{\cmark} & \multicolumn{1}{c|}{\cmark} & \multicolumn{1}{c|}{\cmark} & \multicolumn{1}{c|}{\cmark} \\ \cline{2-10} 
                        & Zhang \textit{et al.} 2021~\cite{zhang2021survey} & \multicolumn{1}{c|}{\cmark} & \multicolumn{1}{c|}{\cmark} & \multicolumn{1}{c|}{\cmark} & \multicolumn{1}{c|}{\xmark} & \multicolumn{1}{c|}{\xmark} & \multicolumn{1}{c|}{\cmark} & \multicolumn{1}{c|}{\cmark} & \multicolumn{1}{c|}{\cmark} \\ \cline{2-10} 
                       & Gupta \textit{et al.}~\cite{gupta2022survey} & \multicolumn{1}{c|}{\cmark} & \multicolumn{1}{c|}{\cmark} & \multicolumn{1}{c|}{\cmark} & \multicolumn{1}{c|}{\xmark} & \multicolumn{1}{c|}{\cmark} & \multicolumn{1}{c|}{\cmark} & \multicolumn{1}{c|}{\cmark} & \multicolumn{1}{c|}{\cmark} \\ \cline{2-10} 
                       & Xu \textit{et al.}~\cite{xu2023asynchronous}                    & \multicolumn{1}{c|}{\cmark} & \multicolumn{1}{c|}{\cmark} & \multicolumn{1}{c|}{\cmark} & \multicolumn{1}{c|}{\xmark} & \multicolumn{1}{c|}{\cmark} & \multicolumn{1}{c|}{\cmark} & \multicolumn{1}{c|}{\cmark} & \multicolumn{1}{c|}{\cmark} \\ \cline{2-10} 
                       &   Li \textit{et al.} 2021~\cite{li2021survey}        & \multicolumn{1}{c|}{\cmark} & \multicolumn{1}{c|}{\cmark} & \multicolumn{1}{c|}{\cmark} & \multicolumn{1}{c|}{\cmark} & \multicolumn{1}{c|}{\cmark} & \multicolumn{1}{c|}{\cmark} & \multicolumn{1}{c|}{\cmark} & \multicolumn{1}{c|}{\cmark} \\ \cline{2-10} 
                       & El \textit{et al.}~\cite{el2022differential}               & \multicolumn{1}{c|}{\xmark} & \multicolumn{1}{c|}{\xmark} & \multicolumn{1}{c|}{\cmark}& \multicolumn{1}{c|}{\xmark} & \multicolumn{1}{c|}{\cmark} & \multicolumn{1}{c|}{\cmark} & \multicolumn{1}{c|}{\xmark} & \multicolumn{1}{c|}{\cmark} \\ \cline{2-10} 
                       &   Kulkarni \textit{et al.}~\cite{kulkarni2020survey} & \multicolumn{1}{c|}{\cmark} & \multicolumn{1}{c|}{\cmark} & \multicolumn{1}{c|}{\xmark} & \multicolumn{1}{c|}{\xmark} & \multicolumn{1}{c|}{\xmark} & \multicolumn{1}{c|}{\cmark} & \multicolumn{1}{c|}{\xmark} & \multicolumn{1}{c|}{\cmark} \\ \cline{2-10} 
                       &  Liu \textit{et al.}\cite{liu2022privacy}             & \multicolumn{1}{c|}{\cmark} & \multicolumn{1}{c|}{\xmark} & \multicolumn{1}{c|}{\cmark} & \multicolumn{1}{c|}{\xmark} & \multicolumn{1}{c|}{\cmark} & \multicolumn{1}{c|}{\cmark} & \multicolumn{1}{c|}{\xmark} & \multicolumn{1}{c|}{\cmark} \\ \cline{2-10} 
                       & Tan \textit{et al.}~\cite{tan2022towards} & \multicolumn{1}{c|}{\xmark} & \multicolumn{1}{c|}{\cmark} & \multicolumn{1}{c|}{\xmark} & \multicolumn{1}{c|}{\xmark} & \multicolumn{1}{c|}{\xmark} & \multicolumn{1}{c|}{\cmark} & \multicolumn{1}{c|}{\xmark} & \multicolumn{1}{c|}{\cmark} \\ \cline{2-10} 
                       & Zhu \textit{et al.} 2021~\cite{zhu2021federated}            & \multicolumn{1}{c|}{\xmark} & \multicolumn{1}{c|}{\cmark} & \multicolumn{1}{c|}{\xmark} & \multicolumn{1}{c|}{\xmark} & \multicolumn{1}{c|}{\xmark} & \multicolumn{1}{c|}{\cmark} & \multicolumn{1}{c|}{\xmark} & \multicolumn{1}{c|}{\cmark} \\ \cline{2-10} 
                       & Ma \textit{et al.}~\cite{ma2022state} & \multicolumn{1}{c|}{\cmark} & \multicolumn{1}{c|}{\cmark} & \multicolumn{1}{c|}{\cmark} & \multicolumn{1}{c|}{\xmark} & \multicolumn{1}{c|}{\xmark} & \multicolumn{1}{c|}{\cmark} & \multicolumn{1}{c|}{\xmark} & \multicolumn{1}{c|}{\cmark} \\ \cline{2-10} 
                       & Aledhari \textit{et al.}~\cite{aledhari2020federated} & \multicolumn{1}{c|}{\cmark} & \multicolumn{1}{c|}{\cmark} & \multicolumn{1}{c|}{\xmark} & \multicolumn{1}{c|}{\xmark} & \multicolumn{1}{c|}{\xmark} & \multicolumn{1}{c|}{\cmark} & \multicolumn{1}{c|}{\cmark} & \multicolumn{1}{c|}{\cmark} \\ \cline{2-10} 
                       &   Kairouz \textit{et al.}~\cite{kairouz2021advances}          & \multicolumn{1}{c|}{\cmark} & \multicolumn{1}{c|}{\cmark} & \multicolumn{1}{c|}{\cmark} & \multicolumn{1}{c|}{\cmark} & \multicolumn{1}{c|}{\cmark} & \multicolumn{1}{c|}{\cmark} & \multicolumn{1}{c|}{\cmark} & \multicolumn{1}{c|}{\cmark} \\ \cline{2-10} 
                       & AbdulRahman \textit{et al.}~\cite{abdulrahman2020survey} & \multicolumn{1}{c|}{\cmark} & \multicolumn{1}{c|}{\cmark} & \multicolumn{1}{c|}{\cmark} & \multicolumn{1}{c|}{\cmark} & \multicolumn{1}{c|}{\xmark} & \multicolumn{1}{c|}{\cmark} & \multicolumn{1}{c|}{\cmark} & \multicolumn{1}{c|}{\cmark} \\ \cline{2-10} 
                       &    Lim \textit{et al.}~\cite{lim2020federated}       & \multicolumn{1}{c|}{\cmark} & \multicolumn{1}{c|}{\cmark} & \multicolumn{1}{c|}{\cmark} & \multicolumn{1}{c|}{\cmark} & \multicolumn{1}{c|}{\xmark} & \multicolumn{1}{c|}{\cmark} & \multicolumn{1}{c|}{\cmark} & \multicolumn{1}{c|}{\cmark} \\ \hline
    \multirow{4}{*}{Healthcare}  &   Xu \textit{et al.}~\cite{xu2021federated}             & \multicolumn{1}{c|}{\cmark} & \multicolumn{1}{c|}{\cmark} & \multicolumn{1}{c|}{\cmark} & \multicolumn{1}{c|}{\xmark} & \multicolumn{1}{c|}{\xmark} & \multicolumn{1}{c|}{\cmark} & \multicolumn{1}{c|}{\cmark} & \multicolumn{1}{c|}{\cmark} \\ \cline{2-10} 
                       & Pfitzner \textit{et al.}\cite{pfitzner2021federated}                  & \multicolumn{1}{c|}{\cmark} & \multicolumn{1}{c|}{\cmark} & \multicolumn{1}{c|}{\cmark} & \multicolumn{1}{c|}{\xmark} & \multicolumn{1}{c|}{\xmark} & \multicolumn{1}{c|}{\cmark} & \multicolumn{1}{c|}{\cmark} & \multicolumn{1}{c|}{\cmark} \\ \cline{2-10} 
                       &           Antunes \textit{et al.}~\cite{antunes2022federated}             & \multicolumn{1}{c|}{\xmark} & \multicolumn{1}{c|}{\cmark} & \multicolumn{1}{c|}{\cmark} & \multicolumn{1}{c|}{\xmark} & \multicolumn{1}{c|}{\xmark} & \multicolumn{1}{c|}{\xmark} & \multicolumn{1}{c|}{\cmark} & \multicolumn{1}{c|}{\cmark} \\ \cline{2-10} 
                       &          Rieke \textit{et al.}~\cite{rieke2020future}             & \multicolumn{1}{c|}{\xmark} & \multicolumn{1}{c|}{\cmark} & \multicolumn{1}{c|}{\cmark} & \multicolumn{1}{c|}{\xmark} & \multicolumn{1}{c|}{\cmark} & \multicolumn{1}{c|}{\cmark} & \multicolumn{1}{c|}{\cmark} & \multicolumn{1}{c|}{\cmark} \\ \hline
    \multirow{4}{*}{IoT}  &   Zhang \textit{et al.} 2022~\cite{zhang2022federated}         & \multicolumn{1}{c|}{\cmark} & \multicolumn{1}{c|}{\cmark} & \multicolumn{1}{c|}{\xmark} & \multicolumn{1}{c|}{\xmark} & \multicolumn{1}{c|}{\xmark} & \multicolumn{1}{c|}{\cmark} & \multicolumn{1}{c|}{\cmark} & \multicolumn{1}{c|}{\cmark} \\ \cline{2-10} 
                       &    Boopalan \textit{et al.}~\cite{boopalan2022fusion}    & \multicolumn{1}{c|}{ \cmark } & \multicolumn{1}{c|}{\cmark} & \multicolumn{1}{c|}{\cmark} & \multicolumn{1}{c|}{\cmark} & \multicolumn{1}{c|}{\cmark}  & \multicolumn{1}{c|}{\cmark} & \multicolumn{1}{c|}{\cmark} & \multicolumn{1}{c|}{\cmark}  \\ \cline{2-10}
                       &    Ramu \textit{et al.}~\cite{ramu2022federated}    & \multicolumn{1}{c|}{ \cmark } & \multicolumn{1}{c|}{\cmark} & \multicolumn{1}{c|}{\cmark} & \multicolumn{1}{c|}{\xmark} & \multicolumn{1}{c|}{\cmark}  & \multicolumn{1}{c|}{\cmark} & \multicolumn{1}{c|}{\cmark} & \multicolumn{1}{c|}{\cmark}  \\ \cline{2-10}
                       &  Du \textit{et al.}~\cite{du2020federated} & \multicolumn{1}{c|}{\cmark} & \multicolumn{1}{c|}{\cmark} & \multicolumn{1}{c|}{\cmark} & \multicolumn{1}{c|}{\cmark} & \multicolumn{1}{c|}{\cmark} & \multicolumn{1}{c|}{\cmark} & \multicolumn{1}{c|}{\cmark} & \multicolumn{1}{c|}{\cmark} \\ \hline
    \multirow{3}{*}{Cybersecurity}  &  Agrawal \textit{et al.}~\cite{agrawal2022federated} & \multicolumn{1}{c|}{\cmark} & \multicolumn{1}{c|}{\cmark} & \multicolumn{1}{c|}{\cmark} & \multicolumn{1}{c|}{\xmark} & \multicolumn{1}{c|}{\cmark} & \multicolumn{1}{c|}{\cmark} & \multicolumn{1}{c|}{\cmark} & \multicolumn{1}{c|}{\cmark} \\ \cline{2-10} 
                       &  Alazab \textit{et al.}~\cite{alazab2021federated}  & \multicolumn{1}{c|}{\xmark} & \multicolumn{1}{c|}{\xmark} & \multicolumn{1}{c|}{\cmark} & \multicolumn{1}{c|}{\xmark} & \multicolumn{1}{c|}{\xmark} & \multicolumn{1}{c|}{\cmark} & \multicolumn{1}{c|}{\cmark} & \multicolumn{1}{c|}{\cmark} \\ \cline{2-10} 
                       &  Ghimire \textit{et al.}~\cite{ghimire2022recent} & \multicolumn{1}{c|}{\cmark} & \multicolumn{1}{c|}{\xmark} & \multicolumn{1}{c|}{\cmark} & \multicolumn{1}{c|}{\xmark} & \multicolumn{1}{c|}{\xmark} & \multicolumn{1}{c|}{\cmark} & \multicolumn{1}{c|}{\cmark} & \multicolumn{1}{c|}{\cmark} \\ \hline
    \multirow{3}{*}{Blockchain}  &  Nguyen \textit{et al.}~\cite{nguyen2021federated} & \multicolumn{1}{c|}{\cmark} & \multicolumn{1}{c|}{\cmark} & \multicolumn{1}{c|}{\cmark} & \multicolumn{1}{c|}{\cmark} & \multicolumn{1}{c|}{\cmark} & \multicolumn{1}{c|}{\cmark} & \multicolumn{1}{c|}{\cmark} & \multicolumn{1}{c|}{\cmark} \\ \cline{2-10} 
                       &  Qu \textit{et al.}~\cite{qu2022blockchain}  & \multicolumn{1}{c|}{\cmark} & \multicolumn{1}{c|}{\cmark} & \multicolumn{1}{c|}{\cmark} & \multicolumn{1}{c|}{\cmark} & \multicolumn{1}{c|}{\cmark} & \multicolumn{1}{c|}{\cmark} & \multicolumn{1}{c|}{\cmark} & \multicolumn{1}{c|}{\cmark} \\ \cline{2-10}
                       &  Zhu \textit{et al.} 2022~\cite{zhu2022blockchain} & \multicolumn{1}{c|}{\cmark} & \multicolumn{1}{c|}{\cmark} & \multicolumn{1}{c|}{\cmark} & \multicolumn{1}{c|}{\cmark} & \multicolumn{1}{c|}{\cmark} & \multicolumn{1}{c|}{\cmark} & \multicolumn{1}{c|}{\cmark} & \multicolumn{1}{c|}{\cmark} \\ \hline
                        %&    & \multicolumn{1}{c|}{\xmark} & \multicolumn{1}{c|}{\xmark} & \multicolumn{1}{c|}{\xmark} & \multicolumn{1}{c|}{\xmark} &  & \multicolumn{1}{c|}{\xmark} & \multicolumn{1}{c|}{\xmark} &  \\ \hline
    \end{tabular}
    \end{table}

\subsection{FL Systems}
\label{sec:flsystems}
Federated learning, with its nature advantages in privacy-preserving decision sharing, has garnered significant attention in both industry and academia, leading to the rapid development of federated learning systems.
The earliest attempt at the large-scale FL system was by Google, where FL was used to improve next-word prediction~\cite{hard2018federated} and query suggestion~\cite{yang2018applied} for Gboard applications.
Subsequently, many novel FL systems have emerged to adapt to diverse federated training scenarios, such as Horizontal FL (e.g., TFF~\cite{abadi2016tensorflow}, FedLab~\cite{zeng2023fedlab}, Felicitas~\cite{zhang2022felicitas}, IBM FL~\cite{ibmfl2020ibm}, OpenFL~\cite{foley2022openfl}), Vertical FL~\cite{wu2022practical} or both (e.g., FATE~\cite{liu2021fate}, FedML~\cite{he2020fedml}, PaddleFL~\cite{ma2019paddlepaddle}, Flower~\cite{beutel2020flower}, FedTree~\cite{li2023fedtree}, NVFLARE~\cite{roth2022nvidia}).
Despite these frameworks covering a wide range of application scenarios, they all follow the server-dominated cooperation mechanism.
This business model restricts FL to function as a collaborative modeling software, rather than an open platform which provides federated training services to the public.

Unlike the FL systems mentioned above, PySyft~\cite{ziller2021pysyft} developed by OpenMined depicts a novel FL cooperation frameworks which is closely realted to our focus. 
PySyft encourages data owners to share their data on a private domain server, which provides data management and privacy controls, as well as limited machine learning analysis APIs for third-party data scientists.
Besides, a public network server will provide connections between data owners and data scientist, enabling datasets search and discovery for platform users.
Recently, a new FL platform named PySyTFF\footnote{https://blog.openmined.org/announcing-proof-of-concept-support-for-tff-in-pysyft-0-7/} was announced. It integrates TFF and PySyft, allowing data scientists to train models under the coordination of TFF and the datasets provided by PySyft domain servers.
However, even with inference controls of datasets, there is still a high security risk associated with exposing access to sensitive data on the Internet~\cite{gamundani2018review}.
To preserve the privacy advantages of FL, in this survey, we investigate open and data-free FL platforms under the scope of model-centric ML~\cite{lou2020towards}.
In such FL platforms, every user is free to collaborate on the training of machine learning models while privacy is protected.

\subsection{As-a-Service Business Model}
\label{sec:aas}
In the current context of Software-as-a-Service (SaaS)~\cite{brereton1999future}, there are several as-a-service cloud computing frameworks that encapsulate ML tasks as services and provides unified APIs for upper layer applications. 
For example, Model-as-a-Service (MaaS)~\cite{geller2007model, roman2009model, zou2012maas, liu2021jizhi, sun2022black} and Machine-Learning-as-a-Service (MLaaS)~\cite{ribeiro2015mlaas, hanzlik2021mlcapsule, hesamifard2018privacy,li2017scaling, kourtellis2020flaas} encapsulate model execution and model development as services.
The original concept of MaaS~\cite{geller2007model, roman2009model} was to provide re-usable and fine-grained user interfaces and visualization tools of domain-specific models (e.g., wealther model, oil spill detection model) for environmental decision support systems.
Subsequently, this concept has been extended to the field of recommendation systems~\cite{zou2012maas} and deep learning based systems~\cite{liu2021jizhi, sun2022black}.
However, in contrast to the focus of this survey, the aforementioned MaaS framework does not involve any user collaboration but solely provides model inference APIs to users.

As the architectures of deep neural networks (DNNs) become increasingly complex, training and maintaining DNNs become more and more challenging~\cite{han2021pre}. 
To address this issue, cloud service providers have introduced MLaaS, which offers an integrated development environment as a service for constructing and operationalizing ML workflows, aiming to reduce the required computational resources.
MLaaS enables users to upload their data for training~\cite{ribeiro2015mlaas, zhao2021veriml, hesamifard2018privacy} or inference~\cite{hanzlik2021mlcapsule}, freeing them from the responsibility of managing hardware resources and implementation.
Most MLaaS providers adopt a pay-by-query business model, such as Google Vertex AI\footnote{https://cloud.google.com/vertex-ai}, Microsoft Azure Machine Learning\footnote{https://azure.microsoft.com/products/machine-learning/} and ChatGPT\footnote{https://chat.openai.com/chat}.
However, privacy protection can be compromised when users upload data to perform training and inference in the cloud.
Moverover, under this model, users are not given the ability to contribute their own models to the repository or collaborate with others to enhance the diversity of available models. 
While there are some ongoing efforts to offer privacy-preserving MLaaS services using techniques such as Trusted Execution Environment (TEE)~\cite{hanzlik2021mlcapsule, mckeen2016intel} and Homomorphic Encryption~\cite{hesamifard2018privacy,gentry2009fully}, it is worth noting that our focus is not solely on privacy.
%Rather, our focus is on promoting a collaborative framework where all entities involved have equal access to services and mutual benefits.

Recently, Kourtellis \textit{et al.}~\cite{kourtellis2020flaas} propose Federated Learning as a Service (FLaaS) that provides high-level and extensible APIs aim to enabling third-party applications to build collaborative, decentralized, privacy-preserving ML models.
Jiang \textit{te al.}~\cite{jiang2022flsys} propose an open FL ecosystem for mobile devices, which shares a similar concept to FLaaS.
However, those approach also follow the traditional server-dominated cooperation framework, which falls under the scope of previous FL surveys\cite{yang2019federated, li2020federated,kairouz2021advances}.

\subsection{Dcentralized FL}
Decentralized FL~\cite{lalitha2018fully, kalra2023decentralized, marfoq2020throughput, hu2019decentralized, sun2022decentralized, shi2023improving}, a novel server-less paradigm of FL, emphasizes the advantages of employing a peer-to-peer model delivery and aggregation network that is free from the dependencies of a central trusted server.
Instead of solely communicating with a central server, participants can fully leverage the network bandwidth by utilizing the network connections between them.
For example, Lalitha \textit{et al.}~\cite{lalitha2018fully} proposed exchange and merge of posterior distribution among neighboring users to collaboratively estimate the global optimal parameter.
Similar to the local training of FedAvg, DFedAvgM~\cite{sun2022decentralized} also suggests that each client communicates with its neighbors after multiple training iterations to improve the convergence rate of training.
On the other hand, ProxyFL~\cite{kalra2023decentralized} improves the privacy of neighbor-wide model sharing by sharing a proxy model through knowledge distillation~\cite{hinton2015distilling}.
However, the major bottleneck lies in the high communication cost of sharing the model with all neighbors in a fully connected network. 

To address this issue, Marfoq \textit{et al.}~\cite{marfoq2020throughput} proposed improving the efficiency of model sharing by selecting a connected subgraph.
Another approach is the use of a gossip-based approach, where model parameters or segments of model parameters are randomly shared with peer neighbors~\cite{hegedHus2021decentralized, hu2019decentralized, shi2023improving}.
Despite the advantages brought by the decentralized design, the training procedure of these frameworks also follows a preset learning task and lacks sustainable cooperation, resulting in a non-public and low reusability FL platform similar to centralized FL.
In fact, our vision for open FL platforms is to extend the FAIR principles~\cite{wilkinson2016fair} for scientific data to the context of machine learning.
We believe that all dedicated models in these platforms should adhere to the principles of being \textbf{Findable, Accessible, Interoperable, and Reusable}.

\begin{comment}
TODO:
given the high scalability of modern edge computing networks, a single MEC server cannot manage to aggregate all updates offloaded from millions of devices.
Therefore, there is an urgent need to develop a more decentralized FL approach without using a central server so as to solve security and scalability issues for enabling the next generation intelligent edge networks.
\end{comment}
% 无中心FL的性能低于传统FL

%\subsection{Blockchain-based FL}
%TODO:

%\subsection{Few-shot FL}
%TODO:

%\subsection{FAIR in FL}
%FAIR Data Principles: Findable, Accessible, Interoperable, Reusable.